
\documentclass[conference]{IEEEtran}
\IEEEoverridecommandlockouts
% The preceding line is only needed to identify funding in the first footnote. If that is unneeded, please comment it out.
%\usepackage{cite}
\usepackage{amsmath,amssymb,amsfonts}
\usepackage{algorithmic}
\usepackage{graphicx}
\usepackage{textcomp}
\usepackage{xcolor}
\def\BibTeX{{\rm B\kern-.05em{\sc i\kern-.025em b}\kern-.08em
    T\kern-.1667em\lower.7ex\hbox{E}\kern-.125emX}}\begin{document}

\title{Research Paper on Disease\\
}


\maketitle

\begin{abstract}
View sample disease research paper. Browse research paper examples for more inspiration. If you need a health research paper written according to all the academic standards, you can always turn to our experienced writers for help. This is how your paper can get an A! Feel free to contact our writing service for professional assistance. We offer high-quality assignments for reasonable rates.
\end{abstract}

\begin{IEEEkeywords}
component, formatting, style, styling, insert
\end{IEEEkeywords}

\section{Abstraction}
Disease is a phenomenon that appears to have struck people globally at all times. However, the conceptions of what disease is have varied with time and place. This research paper gives an overview over various conceptions of disease and highlights what is at stake in the debates on the concept of disease. The core questions for the article are: what is disease and what are the ethical issues entangled in this question?

\section{Introduction}

\subsection{Maintaining the Integrity of the Specifications}

Disease is a phenomenon experienced by most people during their lifetime, and it is something most people fear. Disease is a core concept in the health sciences, in philosophy, and in bioethics, but it is difficult to define. Broadly speaking there are three types of definitions of disease: descriptivist, normativity, and hybrid definitions of disease, claiming that disease is given by phenomena described in nature, by human norms, or both nature and human norms, respectively.

\section{HISTORY AND DEVELOPMENT}
From the interest of understanding and helping people, a wide range of theories and conceptions of disease have emerged. Such theories have altered with time and place. Table 1 gives a brief outline of some theories of disease.

Keep your text and graphic files separate until after the text has been 
formatted and styled. Do not number text heads---{\LaTeX} will do that 
for you.

\subsection{Abbreviations and Acronyms}\label{AA}
Define abbreviations and acronyms the first time they are used in the text, 
even after they have been defined in the abstract. Abbreviations such as 
IEEE, SI, MKS, CGS, ac, dc, and rms do not have to be defined. Do not use 
abbreviations in the title or heads unless they are unavoidable.

\subsection{CONCEPTUAL CLARIFICATION/DEFINITION}
\begin{itemize}
\item Disease implies a right to attention and care, as disease is related to suffering.
\item Disease (in many countries) implies a right to treatment and is thus of great importance to individuals, health professionals, health-care institutions, health insurers, and health policy makers.
\item Disease (in many countries) implies exemptions from duties, such as the duty to work or to take care of others (e.g., relatives or friends).
\item Disease (in many countries) implies a right to economic compensation (e.g., during sick leave) and therefore is important to individuals, employers, insurers, and health policy makers.
\item Disease may exempt from accountability and moral responsibility (in cases of crime).
\item Disease is important for individuals to understand their own situation: “I cannot do or be as I would like, because I am diseased.”
\item Disease is important for individuals to explain  situation to themselves and others.
\item Disease has been important to delimit the tasks of health care from other social tasks and topics.
\item Disease has been important to classify and organize the tasks of health care, e.g., in taxonomies and hospital departments.
\item Disease has been important to delineate the subject matter of health-related sciences.
\end{itemize}

\subsection{CONCEPTUAL CLARIFICATION/DEFINITION}
Descriptivist positions define disease in terms of biological or mental phenomena which can be described in nature (Davies 2003). Hence, such definitions are often also called naturalist definitions. According to the most referred descriptivist definition, disease is an internal condition disturbing natural functioning. Hence, if a bodily or mental function is reduced below what is statistically normal, then there is disease. This definition is oftentimes called “the biostatistical theory of disease,” and it takes into account differences due to gender, age, and species, so that functional differences in such factors do not become diseases (Boorse 1975). That is, a person is not diseased although the person’s heart has reduced functioning at the age of 100 years old compared to the total population. Diseases are kinds that occur in nature, i.e., natural kinds, and they can be classified on the basis of characteristics that can be described in nature.
Be sure that the 
symbols in your equation have been defined before or immediately following 
the equation. Use ``\eqref{eq}'', not ``Eq.~\eqref{eq}'' or ``equation \eqref{eq}'', except at 
the beginning of a sentence: ``Equation \eqref{eq} is . . .''

\subsection{THE ETHICAL DIMENSION OF DISEASE}

Inherent in the debates on the concept of disease, there are a series of ethical issues, such as disease’s inherent imperative to help, over diagnosis, overtreatment, medicalization, and justice. These will be briefly discussed in the following.
Please don't use the \verb|{eqnarray}| equation environment. Use
\verb|{align}| or \verb|{IEEEeqnarray}| instead. The \verb|{eqnarray}|
environment leaves unsightly spaces around relation symbols.

{\LaTeX} can't read your mind. If you assign the same label to a
subsubsection and a table, you might find that Table I has been cross
referenced as Table IV-B3. 

{\LaTeX} does not have precognitive abilities. If you put a
\verb|\label| command before the command that updates the counter it's
supposed to be using, the label will pick up the last counter to be
cross referenced instead. In particular, a \verb|\label| command
should not go before the caption of a figure or a table.

\subsection{THE IMPERATIVE TO HELP}\label{SCM}
\begin{itemize}
\item The word ``data'' is plural, not singular.
\item The subscript for the permeability of vacuum $\mu_{0}$, and other common scientific constants, is zero with subscript formatting, not a lowercase letter ``o''.
\item In American English, commas, semicolons, periods, question and exclamation marks are located within quotation marks only when a complete thought or name is cited, such as a title or full quotation. When quotation marks are used, instead of a bold or italic typeface, to highlight a word or phrase, punctuation should appear outside of the quotation marks. A parenthetical phrase or statement at the end of a sentence is punctuated outside of the closing parenthesis (like this). (A parenthetical sentence is punctuated within the parentheses.)
\item A graph within a graph is an ``inset'', not an ``insert''. The word alternatively is preferred to the word ``alternately'' (unless you really mean something that alternates).
\item Do not use the word ``essentially'' to mean ``approximately'' or ``effectively''.
\item In your paper title, if the words ``that uses'' can accurately replace the word ``using'', capitalize the ``u''; if not, keep using lower-cased.
\item Be aware of the different meanings of the homophones ``affect'' and ``effect'', ``complement'' and ``compliment'', ``discreet'' and ``discrete'', ``principal'' and ``principle''.
\item Do not confuse ``imply'' and ``infer''.
\item The prefix ``non'' is not a word; it should be joined to the word it modifies, usually without a hyphen.
\item There is no period after the ``et'' in the Latin abbreviation ``et al.''.
\item The abbreviation ``i.e.'' means ``that is'', and the abbreviation ``e.g.'' means ``for example''.
\end{itemize}
An excellent style manual for science writers is \cite{b7}.

\subsection{Authors and Affiliations}
\textbf The most obvious ethical aspect of disease is the imperative to help persons who suffer from disease. The term disease indicates that there is something that may be eased. Hence, disease calls us to help persons who are diseased in the best possible manner, either from duty (deontology), in order to maximize the total well-being (consequentialism); from the character of the professional (virtue ethics); or from the calling in the sufferer’s face (proximity ethics)..

\subsection{Identify the Headings}
Headings, or heads, are organizational devices that guide the reader through 
your paper. There are two types: component heads and text heads.

Component heads identify the different components of your paper and are not 
topically subordinate to each other. Examples include Acknowledgments and 
References and, for these, the correct style to use is ``Heading 5''. Use 
``figure caption'' for your Figure captions, and ``table head'' for your 
table title. Run-in heads, such as ``Abstract'', will require you to apply a 
style (in this case, italic) in addition to the style provided by the drop 
down menu to differentiate the head from the text.

Text heads organize the topics on a relational, hierarchical basis. For 
example, the paper title is the primary text head because all subsequent 
material relates and elaborates on this one topic. If there are two or more 
sub-topics, the next level head (uppercase Roman numerals) should be used 
and, conversely, if there are not at least two sub-topics, then no subheads 
should be introduced.

\subsection{Figures and Tables}
\paragraph{Positioning Figures and Tables} Place figures and tables at the top and 
bottom of columns. Avoid placing them in the middle of columns. Large 
figures and tables may span across both columns. Figure captions should be 
below the figures; table heads should appear above the tables. Insert 
figures and tables after they are cited in the text. Use the abbreviation 
``Fig.~\ref{fig}'', even at the beginning of a sentence.



\begin{figure}[htbp]

\caption{Example of a figure caption.}
\label{fig}
\end{figure}

Figure Labels: Use 8 point Times New Roman for Figure labels. Use words 
rather than symbols or abbreviations when writing Figure axis labels to 
avoid confusing the reader. As an example, write the quantity 
``Magnetization'', or ``Magnetization, M'', not just ``M''. If including 
units in the label, present them within parentheses. Do not label axes only 
with units. In the example, write ``Magnetization (A/m)'' or ``Magnetization 
\{A[m(1)]\}'', not just ``A/m''. Do not label axes with a ratio of 
quantities and units. For example, write ``Temperature (K)'', not 
``Temperature/K''.

\section*{Acknowledgment}

The preferred spelling of the word ``acknowledgment'' in America is without 
an ``e'' after the ``g''. Avoid the stilted expression ``one of us (R. B. 
G.) thanks $\ldots$''. Instead, try ``R. B. G. thanks$\ldots$''. Put sponsor 
acknowledgments in the unnumbered footnote on the first page.

\section*{References}

Please number citations consecutively within brackets \cite{b1}. The 
sentence punctuation follows the bracket \cite{b2}. Refer simply to the reference 
number, as in \cite{b3}---do not use ``Ref. \cite{b3}'' or ``reference \cite{b3}'' except at 
the beginning of a sentence: ``Reference \cite{b3} was the first $\ldots$''

Number footnotes separately in superscripts. Place the actual footnote at 
the bottom of the column in which it was cited. Do not put footnotes in the 
abstract or reference list. Use letters for table footnotes.

Unless there are six authors or more give all authors' names; do not use 
``et al.''. Papers that have not been published, even if they have been 
submitted for publication, should be cited as ``unpublished'' \cite{b4}. Papers 
that have been accepted for publication should be cited as ``in press'' \cite{b5}. 
Capitalize only the first word in a paper title, except for proper nouns and 
element symbols.

For papers published in translation journals, please give the English 
citation first, followed by the original foreign-language citation \cite{b6}.

\begin{thebibliography}{00}

%\bibitem{b1} Album D, & Westin, S. (2008), ``Do diseases have a prestige hierarchy? A survey among physicians and medical students. Social Science and Medicine, 66(1), 182–188.
\bibitem{b2} Boorse, C. (1975),On the distinction between disease and illness. Philosophy and Public Affairs, 5, 49–68..
\bibitem{b3} I. S. Jacobs and C. P. Bean, ``Fine particles, thin films and exchange anisotropy,'' in Magnetism, vol. III, G. T. Rado and H. Suhl, Eds. New York: Academic, 1963, pp. 271--350.
\bibitem{b4} K. Elissa, ``Title of paper if known,'' unpublished.
\bibitem{b5} R. Nicole, ``Title of paper with only first word capitalized,'' J. Name Stand. Abbrev., in press.
\bibitem{b6}Sigerist, H. A. (1961). , ``History of Medicine. Vol. II: Early Greek, Hindu, and Persian Medicine. New York: Oxford University Press.
\bibitem{b7}Taylor, F. K. (1979), The concepts of illness, disease and morbus. Cambridge: Cambridge University Press.
\end{thebibliography}


\end{document}