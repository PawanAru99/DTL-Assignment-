\documentclass[12pt]{report}
\usepackage{amsmath}

\usepackage{graphicx}
\author{Pawan Aru}
\title{Biblography Assignment 4}
\begin{document}
\maketitle

~\cite{CALC}Moving Average Crossover: After graphing, two 
moving averages based on separate time periods tend to cross, 
which is known as a moving-average crossover . Without the optional argument label, produces a running number in square brackets
as the label for the reference in the text. The citation numbers are defined by the order in which
the keys appear on the  commands inside “thebibliography” environment, so it is the
responsibility of the student to sort the bibliography entries alphabetically when a bibliography is
created manually. With label, you can give whatever indicator you wish to see when you cite a
reference, i.e. an abbreviation of the author’s name and last two digits of the year. ~\cite{LA}


\begin{thebibliography} {}

\bibitem {LA}Linear Algebra, Introduction to Linear Algebra (2nd edition) by Serge Lang, Springer.

\bibitem{CALC}Calculus for Scientists and Engineers by K.D Joshi, CRC Press..

\end{thebibliography}

\end{document}
